\chapter{Conclusion}

La chaine décisionnelle telle que proposée en introduction est intéressante pour l'exercice mais ne semble pas être la plus optimale pour une utilisation industrialisée en entreprise. \\
Pour la partie TALEND, j'aimerai tester l'intégration du jar générable dans une application JEE ou tout simplement lancé par un ordonnanceur en batch quotidien. La montée en charge du jar généré doit être intéressante à évaluer aussi. Le fait qu'il s'agisse de JAVA rend possible l'investigation par des experts JAVA notamment à l'aide d'outil de profiling. Même si l'expérience utilisateur a été pour moi désastreuse, je ne l'éliminerai pas d'un choix à faire en entreprise sans évaluation et confrontation au besoin.\\
L'utilisation de SAS pour la partie reporting ne m'a pas convaincue. Soit je suis passé à côté, soit l'outil semble dédié à un besoin très particulier qui le ferait exclure de nombreuses études de mise en place d'outil de reporting. L'outil semble destiné à des développeurs SAS ayant surtout besoin d'une capacité à traiter d'éléments statistiques sur de grands volumes de données. Chose que je n'ai pas testée. Je ne pense pas qu'il soit destiné à des décideurs à part évidemment les rapports produits.
