
\chapter{Reporting}
\section{SAS}
\subsection{Présentation}
SAS étant soumis à licence payante, la version testée est la version University Edition. Il s'agit en fait d'un machine virtuelle sur laquelle SAS est installée en version limitée. L'accès se fait via un explorateur pointant vers SASStudio (\url{localhost:10080/}). Cette version limitée limite notamment l'accès aux sources de données. Cette version ne supporte donc pas l'accès à une base de données comme postgresql. Elle ne support que l'accès à des fichiers déposés dans les répertoires partagés de la machine virtuelle. L'entrepôt de données a donc été exporté en csv. Cette modification ne change pas l'aspect conceptuel de cette chaine décisionnelle.
\subsection{Fonctionnement}
Il faut charger tout d'abord les données dans des tables SAS. L'ensemble de ces procédures d'import est disponible en annexe \ref{import.sas}. Lire du csv est facile et l'outil permet une personnalisation assez poussée de ce type d'import.\\
Vient ensuite la préparation des rapports. La partie Data Mart est selon mois représentée dans l'outil par la création d'une requête destinée au rapport voulu (c'est à dire qui répond à la question posée). Pour chaque rapport il a donc été créé une requête dont le résultat est stocké dans des tables. Ces tables sont pour moi les magasins du shéma proposé en introduction.\\ 
Les procédures de création de ces magasins sont en partie données pour exemple en annexe \ref{reporting.sas}
Les rapports sont ici représentés sous forme de graphiques. L'objet n'étant pas ici de tester tous les types de graphiques possibles, seuls des diagrammes en battons sont ici représentés. En annexe \ref{reporting-results} sont donnés quelques uns des diagrammes proposés pour répondre à certaines questions que l'on pourrait se poser sur ces données.\\
L'outil ne semble pas proposer de fonction de cube OLAP telle que le drill down ou drill up. Pour l'exercice, des tables ont été créées dans ce sens en jouant sur les différentes hiérarchies.

\subsection{Conclusions sur l'utilisation de SAS}
SAS semble très puissant au niveau statistiques. SAS ne semble pas proposer d'interface ludique permettant de créer facilement ses rapports autrement que par du code SAS. La notion de cube semble lui être aussi nativement inconnue. Si ce n'est pas lié à la version University edition, cela me le ferait exclure de mes choix pour une utilisation en entreprise destinée à des décideurs. Un composant mobile de la version payante semble plus élaboré mais il faudrait pouvoir le tester. Il faudrait aussi voir SAS sur de gros volumes de données. Peut être que l'outil révèle toute sa puissance alors. Il semble pouvoir être configuré pour se servir d'une grille de calculs ce qui le rend intéressant en fonction du besoin. En tout cas, pour un besoin aussi simple que celui que je propose dans ce rapport, l'outil ne semble pas forcément adapté. I 

\clearpage